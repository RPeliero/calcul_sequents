\documentclass{article}
\usepackage[utf8]{inputenc}
\usepackage{amssymb}
\usepackage{amsmath}
\title{Calcul des séquents}
\date{}
\begin{document}

\maketitle

\section{Notion de séquent}

\subsection{Une définition abstraite}

Un séquent est un couple de la forme
$$\Pi \vdash \Gamma $$

Où $\Pi$ et $\Gamma$ sont des \textit{multi-ensembles de formules logiques}. On parlera de \textit{prémices} pour les éléments de $\Pi$, et de \textit{conclusions} pour les éléments de $\Gamma$

\subsubsection{Une parenthèse sur la notion de multi-ensemble}

En ce qui concerne les ensembles, l'\textit{axiome d'extensionalité} impose que chaque élément n'apparaisse qu'une seule fois.

Dans le cas général, on s'autorise donc à avoir plusieurs occurrences de chaque élément. On parle alors de \textit{multi-ensemble}.

\subsection{Des exemples de séquents}

Soient $A,B,C$ des formules logiques. On a par exemple comme séquent :

1) $\vdash$

2) $\vdash A$

3) $A \vdash $

4) $A \vdash B$

5) $A,B \vdash A$

6) $A,B \vdash A,C$

\subsection{Une formule logique associée au séquent}

On associe au séquent :

$$A_1,A_2,...,A_n \vdash B_1,B_2,...,B_m$$

La formule logique :

$$A_1 \wedge A_2 \wedge ... \wedge A_n \Rightarrow B_1 \lor B_2 \lor ... \lor B_m$$

On prend pour \textbf{True} pour la conjonction vide ($n=0$), et \textbf{False} pour la disjonction vide ($m=0$)

\subsection{Sémantique d'un séquent}

On dit qu'un séquent est \textit{valide} si et seulement si la formule logique qui lui est associée est valide (autrement dit si elle est une tautologie, à savoir sémantiquement équivalente à \textbf{True})
\\

Dans les exemples précédents :

1) $\vdash$ n'est pas valide

2) $\vdash A$ n'est pas valide

3) $A \vdash$ n'est pas valide

4) $A \vdash B$ n'est pas valide

5) $A,B \vdash A$ est valide

6) $A,B \vdash A,C$ est valide

\section{Calcul de séquents}

\subsection{Règles d'inférence}

Pour définir un calcul de séquents, on se donne un ensemble de \textit{règles d'inférence}. Chaque règle est de la forme :

$$\frac{\Pi_1 \vdash \Gamma_1 , \Pi_2 \vdash \Gamma_2 ,..., \Pi_n \vdash \Gamma_n}{\Pi \vdash \Gamma} \; (n \geq 0)$$

Lorsque $n=0$, on dit que la règle est un \textit{axiome}.

\subsection{Le système $\mathcal{G}$}
On va lister les règles d'inférences du système $\mathcal{G}$. Les abréviations qui y sont accolées serviront pour les démonstrations et calculs à venir.

Axiome :
$$\frac{}{\Pi,A \vdash \Gamma,A} \;\;(ax)$$
\\

Règle de coupure :
$$\frac{\Pi \vdash \Gamma, A \;\;\;\;\; \Pi,A \vdash \Gamma}{\Pi \vdash \Gamma} \; \; (cut)$$
\\

\subsubsection{Règles d'inférence logiques}
.

Négation à gauche :

$$\frac{\Pi \vdash \Gamma,A}{\Pi,\neg A \vdash \Gamma} \;\; (\neg g)$$

Elle s'interprète comme suit : "Si $\Pi$ entraîne $A$ ou $\Gamma$, alors, $\Pi$ et $\neg A$ entraînent $\Gamma$. "
\\

Négation à droite :

$$\frac{\Pi,A \vdash \Gamma}{\Pi \vdash \Gamma, \neg A} \;\; (\neg d)$$

"Si $\Pi$ et $A$ entraînent $\Gamma$, alors $\Pi$ entraîne $\Gamma$ ou $\neg A$."
\\

Implication à gauche :

$$\frac{\Pi,A \vdash \Gamma \;\;\;\;\; \Pi,B \vdash \Gamma}{\Pi,A \Rightarrow B \vdash \Gamma} \;\; (\Rightarrow g)$$
\\

Implication à droite :
$$\frac{\Pi,A \vdash B,\Gamma}{\Pi \vdash A \Rightarrow B, \Gamma} \;\; (\Rightarrow d)$$
\\

Conjonction à gauche :
$$\frac{\Pi,A,B \vdash \Gamma}{\Pi,A \wedge B \vdash \Gamma} \;\; (\wedge g)$$
\\

Conjonction à droite :
$$\frac{\Pi \vdash A, \Gamma \;\;\;\; \Pi \vdash B,\Gamma}{\Pi \vdash A \wedge B, \Gamma} \;\; (\wedge d)$$
\\

Disjonction à gauche :
$$\frac{\Pi,A \vdash \Gamma \;\;\;\; \Pi,B \vdash \Gamma}{\Pi, A \lor B \vdash \Gamma} \;\; (\lor g)$$
\\

Disjonction à droite
$$\frac{\Pi \vdash A,B,\Gamma}{\Pi \vdash A \lor B, \Gamma} \;\; (\lor d)$$
\\

\subsection{Réversibilité}

On dit qu'une règle est \textit{réversible} si :
$$\frac{\Pi_1 \vdash \Gamma_1 , \Pi_2 \vdash \Gamma_2 ,..., \Pi_n \vdash \Gamma_n}{\Pi \vdash \Gamma}$$

La formule logique associée à la conclusion $\Pi \vdash \Gamma$ est sémantiquement équivalente à la conjonction des formules logiques des prémisses, $\Pi_i \vdash \Gamma_i$.
\\

On remarque que toutes les règles d'inférences du système $\mathcal{G}$ sont réversibles, à l'exception de la règle de coupure :

En effet, soient $A,B,C$ des formules logiques, en considérant l'instance suivante :
$$\frac{C \vdash B,A \;\;\;\;\; A,C \vdash B}{C \vdash B}$$

On remarque que $[C \Rightarrow (B \lor A)] \wedge [(A \wedge C) \Rightarrow B]$ n'est pas sémantiquement équivalente à $C \Rightarrow B$. En effet :
$$[C \Rightarrow (B \lor A)] \wedge [(A \wedge C) \Rightarrow B] \equiv [\neg C \lor B \lor A] \wedge [\neg (A \wedge C) \lor B]$$

Or :
$$[\neg C \lor B \lor A] \wedge [\neg (A \wedge C) \lor B] \equiv [\neg C \lor B \lor A] \wedge [\neg A \lor \neg C \lor B]$$

Et on sait qu'en développant, on obtient :
$$(A \wedge \neg A) \lor (A \wedge B) \lor (A \wedge \neg C) \lor (B \wedge \neg A) \lor (B \wedge B) \lor (B \wedge \neg C) \lor (\neg C \wedge \neg A) \lor (\neg C \wedge B) \lor (\neg C \wedge \neg C)$$
Autrement dit :
$$(A \wedge B)\lor(A \wedge \neg C)\lor(B \wedge \neg A) \lor B \lor (B \wedge \neg C) \lor \neg C$$

On factorise ce qu'on peut :
$$[A \wedge (B \lor \neg B)] \lor neg C \lor B \lor [\neg C \wedge (A \lor B)]$$

On simplifie :
$$A \lor (C \Rightarrow B) \lor [\neg C \wedge (A \lor B)]$$

Autrement dit, pour $(A,B,C) = \textup{(\textbf{True},\textbf{True},\textbf{False})}$, $C \Rightarrow B$ est fausse, mais $[C \Rightarrow (B \lor A)] \wedge [(A \wedge C) \Rightarrow B]$ est vraie.




































\end{document}
